\section*{sleepy-\/mustache}

Doxygen \href{documentation/html/index.html}{\tt Documentation} is available.

Sleepy mustache is a P\-H\-P framework that comes with solutions for everyday php challenges. All the functionality is optional and tries to be as minimalist as possible.

\subsection*{Included Functionality}


\begin{DoxyItemize}
\item Hooks
\item Templating Engine
\item Singleton P\-D\-O \hyperlink{class_d_b}{D\-B} class
\item Emailing
\item \hyperlink{class_c_s_v}{C\-S\-V} creation
\item Debugging via Output, Email, and \hyperlink{class_d_b}{D\-B}
\item File System Database
\item I\-P 2 Country detection
\item Mobile device detection
\item \hyperlink{class_navigation}{Navigation}
\end{DoxyItemize}

\subsubsection*{Misc}


\begin{DoxyItemize}
\item Robo Caller S\-O\-A\-P A\-P\-I (class.\-robotalker.\-php)
\item S\-Q\-L Select to \hyperlink{class_d_b}{D\-B} Grid (class.\-dbgrid.\-php)
\end{DoxyItemize}

\subsection*{Getting Started}

There are a few globals you will want to set in the include/globals.\-php file.


\begin{DoxyItemize}
\item Setup debugging
\item Set Live site U\-R\-L
\item Set \hyperlink{class_d_b}{D\-B} credentials for live/stage
\item Set Emailing info for live/stage
\item Setup G\-A Account for live/state
\end{DoxyItemize}

\subsection*{Sample Code}

\subsubsection*{Hooks}

The {\itshape Hooks} system is made up of {\itshape hook filters} and {\itshape hook actions}. {\itshape \hyperlink{class_hook}{Hook} actions} are points in the code where you can assign functions to run. For example, we can put a {\itshape hook action} after a record is saved to the database, then assign a function to the {\itshape hook action} that will send an email after the \hyperlink{class_d_b}{D\-B} update. \begin{DoxyVerb}// Save to the database
$db->save();

// add a hook action
$content = Hook::addAction('record_saved');

// Add a function to the hook action
function send_email() {
    // send an email saying a record was updated
}

Hook::doAction(
    'record_saved',
    'send_email'
);
\end{DoxyVerb}


{\itshape \hyperlink{class_hook}{Hook} filters} are similar to {\itshape hook actions} but pass data as parameters to the functions that get assigned to the hook. After manipulating this data you should return the edited data back to the program. \begin{DoxyVerb}// add a hook filter
$content = Hook::addFilter('update_content', $_POST['content']);

// Add a function to the hook filter
function clean_html ($html) {
    $c = htmlentities(trim($html), ENT_NOQUOTES, "UTF-8", false);
    return $c;
}

Hook::applyFilter(
    'update_content',
    'clean_html'
);
\end{DoxyVerb}


The {\itshape modules} directory provides a convenient location to put code that utilized the hooks system. Code inside of the {\itshape modules} directory are automatically added to the program at runtime.

\subsubsection*{Templating}

Templates reside inside the $\ast$'/templates/'$\ast$ folder and should end in a .tpl extension. The templating system works by using placeholders that later get filled in later. The placeholders must have the following syntax\-: \begin{DoxyVerb}{{ placeholder }}
\end{DoxyVerb}


To use a template you instantiate the template class passing in the template name. You then bind data to the placeholders and call the {\itshape \hyperlink{class_template_a2b8e3779f5bd8c38f70307574859bd36}{Template\-::show()}} method. \begin{DoxyVerb}require_once('include/sleepy.php');

$page = new Template('templates/default.tpl');
$page->bind('title', 'Sleepy Mustache');
$page->bind('header', 'Hello world!');
$page->show();
\end{DoxyVerb}


Here is the sample template file (templates/default.\-tpl) \begin{DoxyVerb}<html>
    <head>
        <title>{{ title }}</title>
    </head>
    <body>
        <h1>{{ header }}</h1>
        <p>This page has been viewed {{ hits }} times.</p>
    </body>
</html>
\end{DoxyVerb}


We added a $\ast$\{\{ hits \}\}$\ast$ placeholder in the template above. We can add that functionality using Hooks. \begin{DoxyVerb}// filename: /modules/hits.php
function hook_render_placeholder_topNav() {
    $hits = new FakeClass();

    return $hits->getTotal();
}

Hook::applyFilter(
    'render_placeholder_hits',
    'hook_render_placeholder_hits'
);
\end{DoxyVerb}


The first parameter of {\itshape \hyperlink{class_hook}{Hook}\-:apply\-Filter}, the hook filter, ends in 'hits' which correlates to the name of the placeholder. This hook filter is defined in '{\itshape class.\-template.\-php}'. The second parameter is the name of the function to run when we render the placeholder.

\subsubsection*{Databases}

The database connection settings are defined in the $\ast$/include/global.php$\ast$ file. After the {\itshape L\-I\-V\-E\-\_\-\-U\-R\-L} is set in {\itshape global.\-php} the framework will detect which \hyperlink{class_d_b}{D\-B} to use based on the current U\-R\-L.

To get a database instance, use\-: \begin{DoxyVerb}$db = DB::getInstance();
\end{DoxyVerb}


The \hyperlink{class_d_b}{D\-B} class is static and will automatically handle suppressing multiple instances.

\subsubsection*{Sending emails}

The \hyperlink{class_mailer}{Mailer} class simplifies sending emails by generating headers for you and using an easy to use object to clearly define your email. The \hyperlink{class_mailer}{Mailer} can send emails using an H\-T\-M\-L template or text. \begin{DoxyVerb}$m = new Mailer();
$m->addTo("test@test.com");
$m->addFrom("from.me@test.com");
$m->addSubject("This is a test, don't panic.");
$m->fetchHTML("http://test.com/template.php");
// OR
$m->msgText("This is my message.")
$m->send();
\end{DoxyVerb}


\subsubsection*{\hyperlink{class_c_s_v}{C\-S\-V}}

The \hyperlink{class_c_s_v}{C\-S\-V} class ensures that all records are properly escaped and allows you to easily manipulate data inside of a \hyperlink{class_c_s_v}{C\-S\-V} file. \begin{DoxyVerb}$c = new CSV();
$data = array(
    'George',
    'Washington'
);
$c->add($data);

// Saves to the filesystem
$c->save('presidents.csv');

// OR

// Sends the file to the browser, does not save to the filesystem
$c->show();
\end{DoxyVerb}


\subsubsection*{Debugging}

The {\itshape \hyperlink{class_debug}{Debug}} static class allows you to debug on-\/screen, via email, or by logging to a database. \begin{DoxyVerb}$db = DB::getInstance();
Debug::out($db);
\end{DoxyVerb}


\subsubsection*{File System Database (class.\-fsdb.\-php)}

Sometimes using a database is overkill. A simple solution is to use the {\itshape \hyperlink{class_f_s_d_b}{F\-S\-D\-B}}. It is very simple and does not allow complex queries, however it is fast, easy to use and requires no setup, except checking that proper permissions are set. \begin{DoxyVerb}$fruit = new stdClass();

$fruit->name = "Apple";
$fruit->color = "Red";
$fruit->texture = "Crispy";
$fruit->price = 0.50;

$db = new FSDB();

$db->insert('fruit', $fruit);
$data = $db->select('fruit', 'name', 'Banana');
\end{DoxyVerb}


\subsubsection*{Country detection}

{\itshape Country detection} uses the {\itshape \hyperlink{class_f_s_d_b}{F\-S\-D\-B}} to do a quick lookup of the current country. \begin{DoxyVerb}$i = new IP2CO();

$countryCode = $i->getCountryCode($_SERVER['REMOTE_ADDR']);

if ($countryCode != false) {
    echo $countryCode;
} else {
    echo $_SERVER['REMOTE_ADDR'] . "(" . ip2long($_SERVER['REMOTE_ADDR']) . ") Not found in " . $i->getTable($_SERVER['REMOTE_ADDR']) . ".";
}
\end{DoxyVerb}


\subsubsection*{Mobile detection}

Mobile detection is done by comparing the U\-A (user-\/agent) to a list of currently available mobile and tablet U\-A. \begin{DoxyVerb}$md = new MobiDetect();

if ($md->isMobile()) {
    // goto mobile site
}
\end{DoxyVerb}


\subsubsection*{\hyperlink{class_navigation}{Navigation}}

The navigation is generated by from J\-S\-O\-N. It renders the J\-S\-O\-N into a unordered list with some classes added for the current active page. \begin{DoxyVerb}// Add a placeholder in your template
{{ TopNav }}

// Create a php file in */modules/*
require_once('include/class.navigation.php');

// create a function to add to the *hook filter*
function hook_render_placeholder_TopNav() {

    // Page data is passed via JSON
    $topNavData = '{
        "pages": [
            {
                "title": "Nav 1",
                "link": "/nav1/"
            }, {
                "title": "Nav 2",
                "link": "/nav2/",
                "pages": [
                    {
                        "title": "Subnav 1",
                        "link": "/downloads/fpo.pdf",
                        "target": "_blank"
                    }
                ]
            }
        ]
    }';

    $topNav = new Navigation($topNavData);
    $topNav->setCurrent($_SERVER['SCRIPT_NAME']);

    return $topNav->show();
}

Hook::applyFilter(
    'render_placeholder_TopNav',
    'hook_render_placeholder_TopNav'
);\end{DoxyVerb}
 